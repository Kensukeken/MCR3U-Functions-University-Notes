\documentclass{article}
\usepackage{amsmath}
\usepackage{xcolor}

\begin{document}

\title{{Getting Started with Trigonometry}}
\date{December 11}
\author{{Kensukeken}}
\maketitle

\textcolor{black}{1. Use the Pythagorean theorem to determine each unknown side length.}

\textcolor{black}{a) If the sides of the triangle are $a, b,$ and $c$ (hypotenuse), then $c = \sqrt{a^2 + b^2}$}

Let $a = 3$ and $b = 4$ (units, for example),
\[
c = \sqrt{3^2 + 4^2} = \sqrt{9 + 16} = \sqrt{25} = 5
\]

\textcolor{black}{b) Similarly, if the sides of the triangle are $d, e,$ and $f$ (hypotenuse), then $f = \sqrt{d^2 + e^2}$}

Let $d = 5$ and $e = 12$ (units, for example),
\[
f = \sqrt{5^2 + 12^2} = \sqrt{25 + 144} = \sqrt{169} = 13
\]

\textcolor{black}{2. Using the triangles in question 1, determine the sine, cosine, and tangent ratios for each given angle.}

\textcolor{black}{a) $\sin(\text{LA}) = \frac{\text{opposite side}}{\text{hypotenuse}}, \cos(\text{LA}) = \frac{\text{adjacent side}}{\text{hypotenuse}}, \tan(\text{LA}) = \frac{\text{opposite side}}{\text{adjacent side}}$}

Let $\text{LA}$ be the angle opposite side $a$ and adjacent side $b$,
\[
\textcolor{red}{\sin(\text{LA}) = \frac{3}{5}, \quad \cos(\text{LA}) = \frac{4}{5}, \quad \tan(\text{LA}) = \frac{3}{4}}
\]

\textcolor{black}{b) $\sin(\text{LD}) = \frac{\text{opposite side}}{\text{hypotenuse}}, \cos(\text{LD}) = \frac{\text{adjacent side}}{\text{hypotenuse}}, \tan(\text{LD}) = \frac{\text{opposite side}}{\text{adjacent side}}$}

Let $\text{LD}$ be the angle opposite side $d$ and adjacent side $e$,
\[
\textcolor{red}{\sin(\text{LD}) = \frac{5}{13}, \quad \cos(\text{LD}) = \frac{12}{13}, \quad \tan(\text{LD}) = \frac{5}{12}}
\]

\textcolor{black}{3. Using the triangles in question 1, determine each given angle to the nearest degree.}

\textcolor{black}{a) $\angle B = \arcsin\left(\frac{\text{opposite side}}{\text{hypotenuse}}\right)$}

Let $\angle B$ be the angle opposite side $a$ and hypotenuse $c$,
\[
\textcolor{red}{\angle B = \arcsin\left(\frac{3}{5}\right) \approx 36^\circ}
\]
\newpage
\textcolor{black}{b) $\angle F = \arcsin\left(\frac{\text{opposite side}}{\text{hypotenuse}}\right)$}

Let $\angle F$ be the angle opposite side $d$ and hypotenuse $f$,
\[
\textcolor{red}{\angle F = \arcsin\left(\frac{5}{13}\right) \approx 23^\circ}
\]

\textcolor{black}{4. Use a calculator to evaluate to the nearest thousandth.}

\textcolor{black}{a) $\sin(31^\circ) \approx 0.515$}

\textcolor{black}{b) $\cos(70^\circ) \approx 0.342$}

\textcolor{black}{5. Use a calculator to determine $\theta$ to the nearest degree.}

\textcolor{black}{a) $\cos^{-1}(0.3312) \approx 71^\circ$}

\textcolor{black}{b) $\sin^{-1}(0.7113) \approx 45^\circ$}

\textcolor{black}{c) $\tan^{-1}(1.1145) \approx 48^\circ$}

\textcolor{black}{6. Mario is repairing wires on a radio broadcast tower; he is in the basket of a repair truck 40 m from the tower. When he looks up, he estimates an angle of elevation to the top of the tower as $42^\circ$; when he looks down he estimates the angle of depression to the bottom of the tower as $32^\circ$. How high is the tower?}

\[
\textcolor{red}{\text{Height of tower} = 40 \times \tan(42^\circ) + 40 \times \tan(32^\circ) \approx 40 \times 0.900 + 40 \times 0.625 = 36 + 25 = 61 \, \text{m}}
\]

\textcolor{black}{7. On a sunny day, the tower casts a shadow 35.2m long. At the same time, a car parking meter nearby casts a shadow 1.8m long. How high is a 1.3m tall tower meter that casts a shadow?}

\[
\textcolor{red}{\text{Height of tower} = \frac{\text{Shadow of tower}}{\text{Shadow of meter}} \times \text{Height of meter} = \frac{35.2}{1.8} \times 1.3 \approx 25.455 \, \text{m}}
\]

\end{document}
