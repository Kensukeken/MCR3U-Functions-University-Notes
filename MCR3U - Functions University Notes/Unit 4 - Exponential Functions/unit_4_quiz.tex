\documentclass[12pt]{article}
\usepackage{amsmath, amssymb}

\begin{document}
\title{Exponential Quiz}
\author{Hia Al Saleh}
\date{November 24, 2023}
\maketitle

\section*{Instructions}
Answer all questions. Show all your work.

\section*{Exponent Laws}

\subsection*{Question 1}
Simplify the following expression: $3^4 \cdot 3^2$

\subsection*{Question 2}
Evaluate: $\frac{5^3}{5^2}$

\subsection*{Question 3}
Solve for $x$: $2^{x+1} = 16$

\section*{Logarithms}

\subsection*{Question 4}
Evaluate: $\log_2 32$

\subsection*{Question 5}
Solve for $y$: $\log_4 (2y) = 3$

\subsection*{Question 6}
Apply the logarithmic property: $\log_a (bc)$

\section*{Exponential Equations}

\subsection*{Question 7}
Solve for $z$: $4^{2z} = 64$

\subsection*{Question 8}
Simplify the expression: $(2^3)^{-2}$

\subsection*{Question 9}
If $a^2 = 9$, find the value of $a$.

\section*{Bonus Question}

\subsection*{Question 10}
Prove the exponent law: $(a^m)^n = a^{mn}$


\newpage
\section*{Exponential Quiz - Answers}
\section*{Exponent Laws}

\subsection*{Question 1}
\[3^4 \cdot 3^2 = 81 \cdot 9 = 729\]

\subsection*{Question 2}
\[\frac{5^3}{5^2} = \frac{125}{25} = 5\]

\subsection*{Question 3}
\[2^{x+1} = 16\]
\[2^4 = 16\]
\[x + 1 = 4\]
\[x = 3\]

\section*{Logarithms}

\subsection*{Question 4}
\[\log_2 32 = 5\]

\subsection*{Question 5}
\[\log_4 (2y) = 3\]
\[4^3 = 2y\]
\[y = \frac{4^3}{2} = 32\]

\subsection*{Question 6}
\[\log_a (bc) = \log_a b + \log_a c\]

\section*{Exponential Equations}

\subsection*{Question 7}
\[4^{2z} = 64\]
\[4^{2z} = 4^3\]
\[2z = 3\]
\[z = \frac{3}{2}\]

\subsection*{Question 8}
\[(2^3)^{-2} = 2^{-6} = \frac{1}{2^6} = \frac{1}{64}\]

\subsection*{Question 9}
If \(a^2 = 9\), then \(a = 3\) (since \(3^2 = 9\)).

\section*{Bonus Question}

\subsection*{Question 10}
To prove \((a^m)^n = a^{mn}\), we can use the definition of exponents:
\[(a^m)^n = \underbrace{(a^m) \cdot (a^m) \cdot \ldots \cdot (a^m)}_{n \text{ times}} = a^{m + m + \ldots + m} = a^{mn}\]
\newpage
\section*{Instructions}
Answer all questions. Show all your work.

\section*{Exponent Laws}

\subsection*{Question 1}
Simplify the following expression: $2^{3x} \cdot 2^{2x}$

\textbf{Solution:}
\[2^{3x} \cdot 2^{2x} = 2^{5x}\]

\subsection*{Question 2}
Evaluate: $\frac{3^{2x}}{3^x}$

\textbf{Solution:}
\[\frac{3^{2x}}{3^x} = 3^{2x - x} = 3^x\]

\subsection*{Question 3}
Solve for $y$: $4^{2y-1} = 32$

\textbf{Solution:}
\[4^{2y-1} = 32\]
\[2^{4y-2} = 2^5\]
\[4y - 2 = 5\]
\[4y = 7\]
\[y = \frac{7}{4}\]

\section*{Logarithms}

\subsection*{Question 4}
Evaluate: $\log_5 125$

\textbf{Solution:}
\[\log_5 125 = 3\]

\subsection*{Question 5}
Solve for $z$: $2\log_2 z = 6$

\textbf{Solution:}
\[2\log_2 z = 6\]
\[\log_2 z = 3\]
\[z = 2^3 = 8\]

\subsection*{Question 6}
Apply the logarithmic property: $\log_a \left(\frac{1}{b}\right)$

\textbf{Solution:}
\[\log_a \left(\frac{1}{b}\right) = -\log_a b\]

\section*{Exponential Equations}

\subsection*{Question 7}
Solve for $x$: $e^{2x} = 10$

\textbf{Solution:}
\[2x = \ln 10\]
\[x = \frac{\ln 10}{2}\]

\subsection*{Question 8}
Simplify the expression: $\left(\frac{1}{3}\right)^{-2}$

\textbf{Solution:}
\[\left(\frac{1}{3}\right)^{-2} = 3^2 = 9\]

\subsection*{Question 9}
If $a^3 = 27$, find the value of $a$.

\textbf{Solution:}
\[a^3 = 27\]
\[a = \sqrt[3]{27} = 3\]

\section*{Bonus Question}

\subsection*{Question 10}
Prove the logarithmic property: $\log_a (b^n)$

\textbf{Solution:}
\[\log_a (b^n) = n \cdot \log_a b\]

To prove this, we can use the definition of logarithms:
\[\log_a (b^n) = \underbrace{\log_a b + \log_a b + \ldots + \log_a b}_{n \text{ times}} = n \cdot \log_a b\]
\newpage
\section*{Instructions}
Answer all questions. Show all your work.

\section*{Exponent Laws}

\subsection*{Question 1}
Simplify the following expression: $2^{4x} \cdot 3^{2x} \div 2^{2x}$

\textbf{Solution:}
\[2^{4x} \cdot 3^{2x} \div 2^{2x} = 2^{2x} \cdot 3^{2x} = (2 \cdot 3)^{2x} = 6^{2x}\]

\subsection*{Question 2}
Evaluate: $\frac{5^{3x}}{5^{2x+1}}$

\textbf{Solution:}
\[\frac{5^{3x}}{5^{2x+1}} = 5^{3x - (2x + 1)} = 5^{x - 1}\]

\subsection*{Question 3}
Solve for $y$: $2^{y+3} = 16$

\textbf{Solution:}
\[2^{y+3} = 16\]
\[2^{y+3} = 2^4\]
\[y + 3 = 4\]
\[y = 1\]

\section*{Logarithms}

\subsection*{Question 4}
Evaluate: $\log_4 256$

\textbf{Solution:}
\[\log_4 256 = 4^4 = 16\]

\subsection*{Question 5}
Solve for $z$: $3\log_3 z - 2 = 7$

\textbf{Solution:}
\[3\log_3 z - 2 = 7\]
\[\log_3 z = 3\]
\[z = 3^3 = 27\]

\subsection*{Question 6}
Apply the logarithmic property: $\log_a (b \cdot c)$

\textbf{Solution:}
\[\log_a (b \cdot c) = \log_a b + \log_a c\]

\section*{Exponential Equations}

\subsection*{Question 7}
Solve for $x$: $4^{3x-2} = 32$

\textbf{Solution:}
\[4^{3x-2} = 32\]
\[2^{6x-4} = 2^5\]
\[6x - 4 = 5\]
\[6x = 9\]
\[x = \frac{3}{2}\]

\subsection*{Question 8}
Simplify the expression: $\left(\frac{1}{2}\right)^{-3}$

\textbf{Solution:}
\[\left(\frac{1}{2}\right)^{-3} = 2^3 = 8\]

\subsection*{Question 9}
If $a^4 = 81$, find the value of $a$.

\textbf{Solution:}
\[a^4 = 81\]
\[a = \sqrt[4]{81} = 3\]

\section*{Bonus Question}

\subsection*{Question 10}
Prove the exponent law: $(a^m)^n = a^{mn}$

\textbf{Solution:}
\[(a^m)^n = \underbrace{(a^m) \cdot (a^m) \cdot \ldots \cdot (a^m)}_{n \text{ times}} = a^{m + m + \ldots + m} = a^{mn}\]
\end{document}